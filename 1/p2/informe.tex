\documentclass[a4paper,12pt]{article}
\usepackage[spanish]{babel}
\usepackage[utf8]{inputenc}
\usepackage{imakeidx}
\usepackage{graphicx}
\usepackage{float}
\usepackage{amsmath}
\usepackage[backend=bibtex,style=verbose]{biblatex}
\bibliography{bibliography}
\usepackage{csquotes}
\usepackage{tcolorbox}
\usepackage{multirow}
\usepackage{caption}
\begin{document}
\title{Práctica 2 - Cargas Sometidas a Campos Eléctricos y Magnéticos}

\author{Gabriel D'Andrade Furlanetto - XDD204950}
\date{}
\maketitle
\section{Efecto Hall en Semiconductores}

Existen dos tipos de semiconductores, aquellos de huecos y aquellos de electrones. Se utilizará subíndices para notacionar cada uno (con $n$ para electrónes y $d$ para huecos, para negativo y positivo). Empezamos definiendo la velocidad de deriva: 
$$|\vec{v_n}| = \mu_n |\vec{E}| $$
$$|\vec{v_p}| = \mu_p |\vec{E}| $$

Y se define la corriente de arrastre,

$$I = \sigma ES$$
$$\sigma = qn\mu_n + qp\mu_p $$

Donde $\sigma$ es la conductividad de la muestra,  $n$ y $p$ son las concentraciones de electrones y huecos.

Generalmente, la conductividad se reduce al tipo predominante en la muestra:

$$\sigma_n = qN_d\mu_n$$
$$\sigma_p = qN_A\mu_p$$

Para un montaje específico, es posible obtener la Voltaje Hall:
$$V_{HALL} = \frac{R_H I B}{z}$$
$$R_H = \frac{1}{qn} \quad \text{o} \quad R_H = \frac{1}{qp}$$

Donde se denomina $R_H$ el coeficiente de Hall, que se puede trivialmente relacionar con la movilidad y conductividad:
$$\mu_n = -R_H \sigma_n$$
$$\mu_p = R_H \sigma_p$$

Por fin, se puede calcular la resistencia a partir de la conductividad:

$$R = \frac{L}{\sigma h z}$$

Donde $L$ es la longitud, $h$ la altura y $z$ la anchura.

\section{Relación Carga/Masa}
\subsection{Trayectoria Circular}
Suponiendo que tenemos un campo magnético perpendicular a la velocidad. De esta manera, tenemos que el módulo de la fuerza de Lorentz se puede expresar como:
$$F_L = -e v B$$
Como es una fuerza centrípeta, se tiene que:
$$\frac{e}{m} = \frac{v}{r B}$$ 

O sea, es necesario saber la velocidad, radio de la trayectoria y el campo magnético aplicado para determinar la relación carga/masa. 
Esto se hace acelerando el electrón por un potencial conocido ($U_A$). Por relaciones energéticas, se tiene que:
$$v = \sqrt{2\frac{e}{m} U_A}$$

O sea:
$$\frac{e}{m} = \frac{2U_A}{(r B)^2} $$

Para el montaje utilizado (Bobinas de Helmholtz), se puede determinar el módulo del campo magnético en función de su geometría y la corriente que circula por ella:
$$B \left(\frac{4}{5}\right)^{\frac{3}{2}} \mu_0 \frac{N I_H}{R}$$

Donde $N$ es el número de bobinas y $R$ su radio.

\subsection{Deflexión Eléctrica}

La idea de esta sección es igualar la fuerza eléctrica y magnética para que se compensen. De esta manera, es necesario que (todo en módulo):

$$v = \frac{E}{B}$$

Y en esta geometría se puede afirmar que:
$$\frac{e}{m} = \frac{h v^2 d}{U_D L \left(\frac{L}{2} + D \right)}$$

Finalmente, si conocemos el potencial accelerador y la velocidad, se puede determinar la relación por consideraciones energéticas:
$$\frac{e}{m} = \frac{v^2}{2U_A} $$ 

\end{document}
