%Every piece of package I've acumulated over the last years
%
%
\documentclass[a4paper,12pt]{article}
\usepackage[utf8]{inputenc}
\usepackage{imakeidx}
\usepackage{graphicx}
\usepackage{float}
\usepackage{amsmath}
\usepackage[backend=bibtex,style=verbose]{biblatex}
\bibliography{bibliography}
\usepackage{csquotes}
\usepackage{tcolorbox}
\usepackage{multirow}
\usepackage{caption}
\usepackage{afterpage}
%End of packages
%
%
%
\begin{document}

\title{Gyroscope}
\author{Gabriel D'Andrade Furlanetto - XDD204950}
\maketitle
\pagebreak 

\abstract{In this paper, we will analyze the motion of the gyroscope using data collected during University of Salamanca's Mechanics and Waves Laboratory class. We will pay special attention to its rotation and precession, analyze the relation between their frequencies, and then use that data to to calculate the moment of inertia of the principal axis of our system. We will then calculate the moment of inertia by using the gyroscope as a pendulum, and finally compare the two results.}
\section{Introduction}

\subsection{Objectives}

The main objective of this paper is to study the gyroscope and its movement, namely its rotation, precession and nutation. Special attention will be given to the first two, and as such, their respective frequencies will be calculated.

Furthermore, we will calculate the principal moment of inertia of our gyroscope in two different manners: Firstly, using the frequencies of precession and rotation; Then, using a different experimental setup that uses the gyroscope as a pendulum.

\subsection{Theoretical Introduction}
To accurately describe the motion of the gyroscope, a full theory of the rigid solids is necessary. However, anyone who has come in contact with that will know that setting up the Euler Equations and solving them is an extremely non-trivial task.  

\section{Experimental Procedure}

\subsection{Measurements}

\section{Results}

\section{Final Discussion}

\end{document}
