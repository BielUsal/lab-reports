%Every piece of package I've acumulated over the last years
%%%%%%%%%%%%%%%%%%%%%%%%%%%%%%%%%%%%%%%%%%%%%%%%%%%%%%%%%%%%%%%%%%%%%%%%%%%%%%%%%%%%%%%%%%%%%%
\documentclass[a4paper,12pt]{article}
\usepackage[utf8]{inputenc}
\usepackage{imakeidx}
\usepackage{graphicx}
\usepackage{float}
\usepackage{amsmath}
\usepackage[backend=bibtex,style=verbose]{biblatex}
\bibliography{bibliography}
\usepackage{csquotes}
\usepackage{tcolorbox}
\usepackage{multirow}
\usepackage{caption}
\usepackage{afterpage}
\usepackage[margin=1in]{geometry}
%%%%%%%%%%%%%%%%%%%%%%%%%%%%%%%%%%%%%%%%%%%%%%%%%%%%%%%%%%%%%%%%%%%%%%%%%%%%%%%%%%%%%%%%%%%%%%
\begin{document}

\title{Medida de la Presión Atmosférica}
\author{Gabriel D'Andrade Furlanetto}
\maketitle

\section{Introducción}
\subsection{Datos de Laboratorio}
\begin{table}[h!]
  \centering
  \caption{Temperatura y presión en el laboratorio}
  \begin{tabular}{|c|c|c|c|}
  \hline
    $T_0$(°C) & $T_f$(°C) & $P_0$(mmHg) & $P_f$(mmHg) \\ 
    \hline
    $18.5 \pm 0.5$&$18.0\pm 0.5$ & $702 \pm 1$&$ 702 \pm 1$ \\ 
    \hline 
  \end{tabular}
\end{table}
\subsection{Objetivos}
En está practica, queremos estimar la presión atmosférica en el laboratório, utilizando un montaje experimental que se basa, principalmente, en considerar el aire como un gás ideal y hacerlo pasar por procesos isotermos.

\subsection{Ecuaciones Fundamentales}
En una expansión isoterma de un gás ideal, sabemos que la presión y el volumen del gás son inversamente proporcionales, esto es, si un proceso isotermo nos lleva de $(P_1,V_1)$ a $(P_2,V_2)$, se verifica que:

\begin{equation}
  \label{boyle}
  P_1V_1 = P_2 V_2
\end{equation}

Concretamente, en nuestro experimento, vamos de un estado de $(P_a - P_g, V_1)$ a un estado de $(P_a,V_2)$, donde $P_a$ es la presión atmosférica, $P_m = \frac{Mg}{A}$ la presión que la masa que ponemos ejerce. Aplicando la ecuación \eqref{boyle} y haciendo sencillas manipulaciones algebraicas, tendremos al final que:

\begin{equation}
  \frac{Mg}{A} \frac{V_1}{V_1-V_2}
\end{equation}

\section{Procedimiento Experimental}

\subsection{Datos Experimentales}
Haciendo las mediciones, obtenemos los siguientes valores:

$$$$

\subsection{Cálculo de Errores}

\section{Resultados}

\section{Conclusiones}

\subsection{Observaciones y sugerencias}

\subsection{Measurements}

\section{Results}

\section{Final Discussion}

\printbibliography
\end{document}

