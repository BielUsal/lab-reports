%Every piece of package I've acumulated over the last years
%%%%%%%%%%%%%%%%%%%%%%%%%%%%%%%%%%%%%%%%%%%%%%%%%%%%%%%%%%%%%%%%%%%%%%%%%%%%%%%%%%%%%%%%%%%%%%
\documentclass[a4paper,12pt]{article}
\usepackage[utf8]{inputenc}
\usepackage{imakeidx}
\usepackage{graphicx}
\usepackage{float}
\usepackage{amsmath}
\usepackage[backend=bibtex,style=verbose]{biblatex}
\bibliography{bibliography}
\usepackage{csquotes}
\usepackage{tcolorbox}
\usepackage{multirow}
\usepackage{caption}
\usepackage{afterpage}
\usepackage[margin=1in]{geometry}
%%%%%%%%%%%%%%%%%%%%%%%%%%%%%%%%%%%%%%%%%%%%%%%%%%%%%%%%%%%%%%%%%%%%%%%%%%%%%%%%%%%%%%%%%%%%%%
\begin{document}

\title{Ecuación empírica de estado de gases ideales}
\author{\underline{Gabriel D'Andrade Furlanetto}, Sergio Bernardo Pérez} 
\maketitle

\newpage
\thispagestyle{empty}
\mbox{}
\newpage 


\section{Datos de Laboratorio}
\begin{table}[H]
\centering
  \centering
  \caption{Temperatura y presión en el laboratorio}
  \label{bar}
  \begin{tabular}{|c|c|c|c|}
  \hline
    $T_0$(°C) & $T_f$(°C) & $P_0$(mmHg) & $P_f$(mmHg) \\ 
    \hline
    $17\pm 0.5$&$17\pm 0.5$ & $694 \pm 1$&$ 694 \pm 1$ \\ 
    \hline 
  \end{tabular}
\end{table}

\section{Objetivos}
Comprobar que el aire a presiones y temperaturas normales se comporta como un gás ideal y, por lo tanto, verifica las leyes de Boyle-Mariotte y Gay-Lussac.
\section{Ecuaciones Fundamentales}
Cuando un gas ideal realiza un proceso isotermo, se verifica que la presión y el volumen son inversamente proporcionales, la ley de Boyle-Mariotte:
\begin{equation}
  PV = K_1
\end{equation}

Cuando realiza un proceso isobaro, se verifica la temperatura y el volumen son proporcionales, la primera ley de Gay-Lussac:

\begin{equation}
  \frac{V}{T} = K_2
\end{equation}

Y, finalmente, en un proceso isocoro, la presión y temperatura son proporcionales, o sea, se verifica la segunda ley de Gay-Lussac:

\begin{equation}
  \frac{P}{T} = K_3
\end{equation}

Que se pueden combinar para forma la ecuación del gas ideal:

\begin{equation}
  PV = nRT 
\end{equation}

Donde $n$ es el número de moles, y $R = 8.314 \text{ J/(mol K)}$ es una constante de proporcionalidad, la constante universal de los gases. De esa ecuación, podemos calcular los coeficientes termomecánicos para un gas ideal:
\begin{equation}
  \alpha = \frac{1}{V}\left(\frac{\partial V}{\partial T}\right)_P = \frac{1}{T}
\end{equation}

\begin{equation}
  \kappa_T = -\frac{1}{V}\left(\frac{\partial V}{\partial P}\right)_T = \frac{1}{P}
\end{equation}

\begin{equation}
  \beta = \frac{1}{p}\left(\frac{\partial P}{\partial T}\right)_V = \frac{1}{T}
\end{equation}
\section{Datos Experimentales}
\begin{table}[H]
\centering
  \caption{Datos experimentales}
  \begin{tabular}{|cc|cc|}
    \hline
    \multicolumn{2}{|c|}{Isoterma}                                    & \multicolumn{2}{c|}{Isocora}                                \\ \hline
    \multicolumn{2}{|c|}{$t = 20.4\pm 1$}                             & \multicolumn{2}{c|}{$V = 30.0 \pm 0.5  cm^3$}               \\ \hline
    \multicolumn{1}{|c|}{V $(cm^3)$}             & P (hPa)            & \multicolumn{1}{c|}{$t$(°C)}           & P(hPa)             \\ \hline
    \multicolumn{1}{|c|}{30}                     & 996                & \multicolumn{1}{c|}{20.4}              & 996                \\ \hline
    \multicolumn{1}{|c|}{35}                     & 862                & \multicolumn{1}{c|}{29.1}              & 1040               \\ \hline
    \multicolumn{1}{|c|}{40}                     & 757                & \multicolumn{1}{c|}{40.1}              & 1077               \\ \hline
    \multicolumn{1}{|c|}{45}                     & 672                & \multicolumn{1}{c|}{49.4}              & 1092               \\ \hline
    \multicolumn{1}{|c|}{50}                     & 603                & \multicolumn{1}{c|}{59}                & 1115               \\ \hline
    \multicolumn{1}{|c|}{55}                     & 550                & \multicolumn{1}{c|}{}                  &                    \\ \hline
    \multicolumn{1}{|c|}{60}                     & 497                & \multicolumn{1}{c|}{}                  &                    \\ \hline
    \multicolumn{1}{|c|}{$\Delta V = 0.5 cm^3 $} & $\Delta P = 1 hPa$ & \multicolumn{1}{c|}{$\Delta t =  0.1$} & $\Delta P = 1 hPa$ \\ \hline
  \end{tabular}
\end{table}
\section{Datos para la representación gráfica}
\begin{table}[H]
  \centering
  \caption{Datos para las representaciones de la isoterma}
  \begin{tabular}{|ccccc|}
    \hline
    \multicolumn{5}{|c|}{$T = 293.4 \pm 0.1$ K}                                                                                                     \\ \hline
    \multicolumn{2}{|c|}{$x = P (Pa)$}                          & \multicolumn{2}{c|}{$\Delta y (mol)$}                     & $\Delta z (m^{-3})$ \\ \hline
    \multicolumn{1}{|c|}{99600} & \multicolumn{1}{c|}{1.22$\times 10^{-3}$} & \multicolumn{1}{c|}{0.020$\times 10^{-3}$} & \multicolumn{1}{c|}{33333} & 556                 \\ \hline
    \multicolumn{1}{|c|}{86200} & \multicolumn{1}{c|}{1.24$\times 10^{-3}$} & \multicolumn{1}{c|}{0.018$\times 10^{-3}$} & \multicolumn{1}{c|}{28571} & 408                 \\ \hline
    \multicolumn{1}{|c|}{75700} & \multicolumn{1}{c|}{1.24$\times 10^{-3}$} & \multicolumn{1}{c|}{0.016$\times 10^{-3}$} & \multicolumn{1}{c|}{25000} & 313                 \\ \hline
    \multicolumn{1}{|c|}{67200} & \multicolumn{1}{c|}{1.24$\times 10^{-3}$} & \multicolumn{1}{c|}{0.014$\times 10^{-3}$} & \multicolumn{1}{c|}{22222} & 247                 \\ \hline
    \multicolumn{1}{|c|}{60300} & \multicolumn{1}{c|}{1.24$\times 10^{-3}$} & \multicolumn{1}{c|}{0.013$\times 10^{-3}$} & \multicolumn{1}{c|}{20000} & 200                 \\ \hline
    \multicolumn{1}{|c|}{55000} & \multicolumn{1}{c|}{1.24$\times 10^{-3}$} & \multicolumn{1}{c|}{0.012$\times 10^{-3}$} & \multicolumn{1}{c|}{18182} & 165                 \\ \hline
    \multicolumn{1}{|c|}{49700} & \multicolumn{1}{c|}{1.22$\times 10^{-3}$} & \multicolumn{1}{c|}{0.010$\times 10^{-3}$} & \multicolumn{1}{c|}{16667} & 139                 \\ \hline
  \end{tabular}
\end{table}

\begin{table}[H]
\centering
  \caption{Datos para la representación gráfica de la isocora}
  \begin{tabular}{|c|c|}
    \hline
    \multicolumn{2}{|c|}{$V = 30 \pm 0.5$}         \\\hline
    $x = T \text{ (K)}$  & $y = P \text{ (Pa)} $ \\\hline
    293.4                & 99600                 \\\hline
    302.1                & 104000                \\\hline
    313.1                & 107700                \\\hline
    322.4                & 109200                \\\hline
    332                  & 111500                \\\hline
    \multicolumn{1}{|c|}{} & $\Delta y = 100 Pa$\\ \hline  
  \end{tabular}
\end{table}

\section{Representación gráfica}
\pagebreak

\section{Ajuste de datos}
Para el cálculo, utilizaremos que, en general, para una regresión del tipo $y = ax + b$, se tiene que:

\begin{equation}
  a = \frac{n\sum x y - \sum x \sum y}{n \sum x^2 - \left(\sum x\right)^2}
\end{equation}
\begin{equation}
  b = \frac{\sum y \sum x^2 - \sum x \sum x y}{n\sum x^2 - \left(\sum x \right)^2}
\end{equation}

\begin{equation}
R^2=\frac{\left(\sum x y -\sum x \sum y\right)^2}{\left(n\sum x^2 - \left(\sum x\right)^2\right)\left(n\sum y^2 - \left(\sum y\right)^2\right)} 
\end{equation}

\begin{table}[H]
  \centering
  \caption{Datos para el ajuste de $\frac{PV}{RT}$ frente a $P$}
  \label{t1}
  \begin{tabular}{|c|c|c|c|c|c|}
    \hline
    $n$ & $\sum x$ & $\sum y$ & $\sum x^2$  & $\sum y^2$ & $\sum xy$   \\ \hline
    7   & 493700   & 8.64E-03 & 36728110000 & 1.07E-05   & 609.3841188 \\ \hline
  \end{tabular}
\end{table}
Recta de regresión $\frac{PV}{RT} = a_1 P + b_1 $:
\begin{equation}
  a_1 = -3.96 \times 10^{-11} \frac{mol}{Pa}
\end{equation}
\begin{equation}
   b_1 = 1.24 \times 10^{-3} mol
\end{equation}
\begin{equation}
  R_1^2 = 0.00857
\end{equation}

\begin{table}[H]
  \centering
  \caption{Datos para el ajuste de $\frac{1}{V}$ frente a $P$}
  \begin{tabular}{|c|c|c|c|c|c|}
    \hline
    $n$ & $\sum x$ & $\sum z$ & $\sum x^2$   & $\sum z^2$ & $\sum xz$   \\ \hline
    5   & 725900   & 1.36E-02 & 257386710000 & 8.08E-05   & 4553.016987 \\ \hline
  \end{tabular}
\end{table}

Recta de regresión $\frac{1}{V} = a_2 P + b_2 $:

\begin{equation}
  a_2 = 0.334 \frac{1}{m^3 Pa}
\end{equation}
\begin{equation}
   b_2 = -159 \frac{1}{m^3}
\end{equation}
\begin{equation}
  R_2^2 = 0.00857
\end{equation}

\begin{table}[H]
  \centering
  \caption{Datos para el ajuste de $P$ frente a $T$}
  \begin{tabular}{|c|c|c|c|c|c|}
    \hline
    $n$ & $\sum x$ & $\sum y$ & $\sum x^2$ & $\sum y^2$  & $\sum xy$ \\ \hline
    5   & 1563     & 532000   & 489545.34  & 56692340000 & 166585990 \\ \hline
  \end{tabular}
\end{table}

Recta de regresión $P = a_3 T + b_3 $:

\begin{equation}
  a_3 = 297.19 \frac{Pa}{K}
\end{equation}
\begin{equation}
   b_3 = 13497 Pa
\end{equation}
\begin{equation}
  R_3^2 = 0.960
\end{equation}

\section{Cálculo de Errores}
En general, para regresiones lineales, tenemos dos de error, el estadístico y el instrumental. El primero, se calcula con:

$$\Delta_{est} a= a\sqrt{\frac{R_1^{-2}-1}{n-2}}$$ 
$$\Delta_{est} b = \Delta_{est} a\sqrt{\frac{\sum_i x_i^2}{n}}$$

Y el segundo se calcula con:

  $$\Delta_{inst} a = \sqrt{\sum_j\left[ \frac{n\cdot x_j - \sum_ix_i}{n\sum_i x_i^2-\left(  \sum_i  x_i\right)^2} \right]^2 \cdot \Delta y_j^2}$$ 

  $$\Delta_{inst} b =\sqrt{\sum_j\left[ \frac{\sum x_i^2 -x_j \sum_i x_i}{n\sum_i x_i^2-\left(  \sum_i  x_i\right)^2} \right]^2 \cdot \Delta y_j^2}$$

Finalmente, se puede calcular el error total por la regla de cuadratura:

\begin{equation}
  \Delta a = \sqrt{\Delta_{est} a^2 + \Delta_{inst} a^2}
\end{equation}
\begin{equation}
  \Delta b = \sqrt{\Delta_{est} b^2 + \Delta_{inst} b^2}
\end{equation}

Entonces, tenemos que:


  $$\Delta_{est} a_1 = 1.90 \times 10^{-10} \frac{mol}{Pa}$$
  $$\Delta_{esc} a_1 = 3.78 \times 10^{-10} \frac{mol}{Pa}$$
  $$\Delta_{est} b_1 = 1.38 \times 10^{-5} mol $$
  $$\Delta_{esc} b_1 = 2.50 \times 10^{-5} mol$$

  \begin{equation}
    \Delta a_1 = 4.23 \times 10^{-10}\frac{mol}{Pa}
  \end{equation}
  \begin{equation}
    \Delta b_1 = 2.85\times 10^{-5} mol
  \end{equation}



  $$\Delta_{est} a_2 = 0.0035 \frac{1}{m^3 Pa} $$
  $$\Delta_{esc} a_2 = 0.0094 \frac{1}{m^3 Pa} $$
  $$\Delta_{est} b_2= 257  \frac{1}{m^3}$$
  $$\Delta_{esc} b_2 = 587 \frac{1}{m^3}$$

  \begin{equation}
    \Delta a_2 = 0.01 \frac{1}{m^3 Pa}
  \end{equation}
  \begin{equation}
    \Delta b_2 = 640 \frac{1}{m^3}
  \end{equation}

  $$\Delta_{est} a_3 = 35\frac{T}{ Pa}   $$
  $$\Delta_{esc} a_3 = 3.2418 \frac{T}{ Pa} $$
  $$\Delta_{est} b_3= 10951  T$$
  $$\Delta_{esc} b_3 = 1014 T$$

  \begin{equation}
    \Delta a_3 = 35.1 \frac{1}{m^3 Pa}
  \end{equation}
  \begin{equation}
    \Delta b_3 = 10951 \frac{1}{m^3}
  \end{equation}

Finalmente, no queremos los coeficientes directamente pero sí $n$, $\kappa_{T}$ a $P = 100000 atm$ y $\beta$ a $300K$. De esa manera, calculamos primero a $n$:

\begin{equation}
  n = b_1 = 1.24 \times 10^{-3} \text{ mol}
\end{equation}
Por lo que tenemos, trivialmente, que
\begin{equation}
  \Delta n = \Delta b_1 = 0.029 \times 10^{-3} \text{ mol}
\end{equation}

Para $\kappa_T$, tenemos que:
\begin{equation}
   \kappa_T = V \left(\frac{\partial 1/V}{\partial P}\right) = \frac{1}{a_2 P + b_2 }a_2 = 1.00 \times 10^{-5}
\end{equation} 

\begin{equation}
  \Delta \kappa_T = \sqrt{\left(\frac{\partial \kappa_T }{\partial a_2} \Delta a_2\right)^2 + \left(\frac{\partial \kappa_T }{\partial b_2} \Delta b_2\right)^2} = 0.02 \times 10^{-5}
\end{equation}

Finalmente, para $\beta$, tendremos que

\begin{equation}
   \beta = \frac{1}{P}\left(\frac{\partial P}{\partial T}\right) = \frac{1}{a_3 T + b_3 }a_3 = 2.9 \times 10^{-3} K^{-1}
\end{equation} 

\begin{equation}
  \Delta \beta = \sqrt{\left(\frac{\partial \beta }{\partial a_3} \Delta a_3\right)^2 + \left(\frac{\partial \beta }{\partial b_3} \Delta b_3\right)^2} = 0.3 \times 10^{-3} 
\end{equation}

\section{Resultados}
\begin{tcolorbox}
  \begin{equation}
    n = \left(1.24 \pm 0.03\right) \times 10^{-3} \text{mol}
  \end{equation}
  \begin{equation}
    \kappa_T = \left(1.00 \pm 0.02\right) \times 10^{-5} \text{Pa}^{-1}
  \end{equation}
  \begin{equation}
    \beta = \left(2.9 \pm 0.3\right) \times 10^{-3} \text{K}^{-1}
  \end{equation}
\end{tcolorbox}
\section{Conclusiones}

\section{Observaciones y sugerencias}


\end{document}

