%Every piece of package I've acumulated over the last years
%%%%%%%%%%%%%%%%%%%%%%%%%%%%%%%%%%%%%%%%%%%%%%%%%%%%%%%%%%%%%%%%%%%%%%%%%%%%%%%%%%%%%%%%%%%%%%
\documentclass[a4paper,12pt]{article}
\usepackage[utf8]{inputenc}
\usepackage{imakeidx}
\usepackage{graphicx}
\usepackage{float}
\usepackage{amsmath}
\usepackage[backend=bibtex,style=verbose]{biblatex}
\bibliography{bibliography}
\usepackage{csquotes}
\usepackage{tcolorbox}
\usepackage{multirow}
\usepackage{caption}
\usepackage{afterpage}
\usepackage[margin=1in]{geometry}
%%%%%%%%%%%%%%%%%%%%%%%%%%%%%%%%%%%%%%%%%%%%%%%%%%%%%%%%%%%%%%%%%%%%%%%%%%%%%%%%%%%%%%%%%%%%%%
\begin{document}

\title{}
\author{\underline{Gabriel D'Andrade Furlanetto}, Sergio Bernardo Pérez} 
\maketitle

\newpage
\thispagestyle{empty}
\mbox{}
\newpage 


\section{Datos de Laboratorio}
\begin{table}[h!]
  \centering
  \caption{Temperatura y presión en el laboratorio}
  \label{bar}
  \begin{tabular}{|c|c|c|c|}
  \hline
    $T_0$(°C) & $T_f$(°C) & $P_0$(mmHg) & $P_f$(mmHg) \\ 
    \hline
    $17\pm 0.5$&$17\pm 0.5$ & $694 \pm 1$&$ 694 \pm 1$ \\ 
    \hline 
  \end{tabular}
\end{table}


\section{Objetivos}

\section{Ecuaciones Fundamentales}


\section{Datos Experimentales}

\section{Datos para la representación gráfica}

\section{Representación gráfica}


\section{Ajuste de datos}
Para el cálculo, utilizaremos que, en general, para una regresión del tipo $y = ax + b$, se tiene que:

\begin{equation}
  a = \frac{n\sum x y - \sum x \sum y}{n \sum x^2 - \left(\sum x\right)^2}
\end{equation}
\begin{equation}
  b = \frac{\sum y \sum x^2 - \sum x \sum x y}{n\sum x^2 - \left(\sum x \right)^2}
\end{equation}

\begin{equation}
R^2=\frac{\left(\sum x y -\sum x \sum y\right)^2}{\left(n\sum x^2 - \left(\sum x\right)^2\right)\left(n\sum y^2 - \left(\sum y\right)^2\right)} 
  
\end{equation}
\section{Cálculo de Errores}

En general, para regresiones lineales, tenemos dos de error, el estadístico y el instrumental. El primero, se calcula con:

$$\Delta_{est} a= a\sqrt{\frac{R_1^{-2}-1}{n-2}}$$ 
$$\Delta_{est} b = \Delta_{est} a\sqrt{\frac{\sum_i x_i^2}{n}}$$

Y el segundo se calcula con:

  $$\Delta_{inst} a = \sqrt{\sum_j\left[ \frac{n\cdot x_j - \sum_ix_i}{n\sum_i x_i^2-\left(  \sum_i  x_i\right)^2} \right]^2 \cdot \Delta y_j^2}$$ 

  $$\Delta_{inst} b =\sqrt{\sum_j\left[ \frac{\sum x_i^2 -x_j \sum_i x_i}{n\sum_i x_i^2-\left(  \sum_i  x_i\right)^2} \right]^2 \cdot \Delta y_j^2}$$

Finalmente, se puede calcular el error total por la regla de cuadratura:

\begin{equation}
  \Delta a = \sqrt{\Delta_{est} a^2 + \Delta_{inst} a^2} = 72.5\text{K}
\end{equation}
\begin{equation}
  \Delta b = \sqrt{\Delta_{est} b^2 + \Delta_{inst} b^2} = 0.28
\end{equation}
\section{Resultados}
\begin{tcolorbox}
  
\end{tcolorbox}
\section{Conclusiones}

\section{Observaciones y sugerencias}


\printbibliography
\end{document}

