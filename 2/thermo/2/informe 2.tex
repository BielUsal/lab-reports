%Every piece of package I've acumulated over the last years
%%%%%%%%%%%%%%%%%%%%%%%%%%%%%%%%%%%%%%%%%%%%%%%%%%%%%%%%%%%%%%%%%%%%%%%%%%%%%%%%%%%%%%%%%%%%%%
\documentclass[a4paper,12pt]{article}
\usepackage[utf8]{inputenc}
\usepackage{imakeidx}
\usepackage{graphicx}
\usepackage{float}
\usepackage{amsmath}
\usepackage[backend=bibtex,style=verbose]{biblatex}
\bibliography{bibliography}
\usepackage{csquotes}
\usepackage{tcolorbox}
\usepackage{multirow}
\usepackage{caption}
\usepackage{afterpage}
\usepackage[spanish]{babel}
\decimalpoint
\usepackage[margin=1in]{geometry}
%%%%%%%%%%%%%%%%%%%%%%%%%%%%%%%%%%%%%%%%%%%%%%%%%%%%%%%%%%%%%%%%%%%%%%%%%%%%%%%%%%%%%%%%%%%%%%
\begin{document}

\title{Transición de fase líquido-vapor. Punto crítico.}
\author{\underline{Gabriel D'Andrade Furlanetto}, Sergio Bernardo Pérez} 
\maketitle

\pagebreak 

\pagebreak
\section{Datos de Laboratorio}
\begin{table}[h!]
  \centering
  \caption{Temperatura y presión en el laboratorio}
  \label{bar}
  \begin{tabular}{|c|c|c|c|}
  \hline
    $T_0$(°C) & $T_f$(°C) & $P_0$(mmHg) & $P_f$(mmHg) \\ 
    \hline
    $16.5 \pm 0.5$&$17.5\pm 0.5$ & $697 \pm 1$&$ 697 \pm 1$ \\ 
    \hline 
  \end{tabular}
\end{table}

\section{Objetivos}
En esta práctica, queremos determinar algunas isotermas del hexafluoruro de azufre: Una subcrítica, una crítica y otra supercrítica. Con eso, determinaremos también su punto crítico y su calor latente molar de vaporización. 
\section{Ecuaciones Fundamentales}
Para calcular el calor latente molar de vaporización, necesitamos mirar la curva de coexistencia líquido-vapor, que estará descrita por la ecuación de Clapeyron-Clausius:

\begin{equation}
\label{clapclau}
\frac{dP}{dT} = \frac{\Delta h_v}{T(v_g-v_l)}  
\end{equation}

Una ecuación que, en general, no es nada trivial de resolver. No obstante, si suponemos que $v_{g} \gg v_l $, que el vapor se aproxima a un gas ideal, y que $\Delta h_{v}$ no depende de presión o temperatura, la ecuación diferencial \eqref{clapclau} se puede integrar, donde obtenemos que:

\begin{equation}
  \label{final}
  \ln\left(\frac{P}{P_0}\right) = \frac{\Delta h_v}{R} \left(\frac{1}{T_0}-\frac{1}{T}\right)
\end{equation}

Donde tomaremos $P_{0} = 1 atm$, y, por eso, $T_{0} = T_b $, que es la temperatura de ebullición normal (A una atmósfera).
\section{Isotermas}
\subsection{Datos Experimentales}
Haciendo cada una de las isotermas, obtenemos los siguientes datos:
\begin{table}[H]
  \caption{Volúmenes y presiones para diferentes isotermas }
    \centering
  \begin{tabular}{|cccccc|}
    \hline
    \multicolumn{6}{|c|}{Isotermas}                                                                                                                                        \\ \hline
    \multicolumn{2}{|c|}{Subcritico}                              & \multicolumn{2}{c|}{Critico}                                 & \multicolumn{2}{c|}{Supercritico}       \\ \hline
    \multicolumn{2}{|c|}{$T=20.5 \pm 0.1$°C}                      & \multicolumn{2}{c|}{$T=46.9 \pm 0.1$ °C}                     & \multicolumn{2}{c|}{$T=49.9\pm 0.1$°C}  \\ \hline
    \multicolumn{1}{|c|}{P(atm)} & \multicolumn{1}{c|}{V($cm^3$)} & \multicolumn{1}{c|}{P(atm)} & \multicolumn{1}{c|}{V($cm^3$)} & \multicolumn{1}{c|}{P(atm)} & V($cm^3$) \\ \hline
    \multicolumn{1}{|c|}{14.3}   & \multicolumn{1}{c|}{3.50}      & \multicolumn{1}{c|}{16.8}   & \multicolumn{1}{c|}{3.50}      & \multicolumn{1}{c|}{15.3}   & 4.00      \\ \hline
    \multicolumn{1}{|c|}{15.3}   & \multicolumn{1}{c|}{3.25}      & \multicolumn{1}{c|}{18.3}   & \multicolumn{1}{c|}{3.25}      & \multicolumn{1}{c|}{16.3}   & 3.75      \\ \hline
    \multicolumn{1}{|c|}{16.3}   & \multicolumn{1}{c|}{3.00}      & \multicolumn{1}{c|}{19.2}   & \multicolumn{1}{c|}{3.00}      & \multicolumn{1}{c|}{17.3}   & 3.50      \\ \hline
    \multicolumn{1}{|c|}{17.3}   & \multicolumn{1}{c|}{2.75}      & \multicolumn{1}{c|}{20.7}   & \multicolumn{1}{c|}{2.75}      & \multicolumn{1}{c|}{18.8}   & 3.25      \\ \hline
    \multicolumn{1}{|c|}{18.5}   & \multicolumn{1}{c|}{2.50}      & \multicolumn{1}{c|}{22.5}   & \multicolumn{1}{c|}{2.50}      & \multicolumn{1}{c|}{19.7}   & 3.00      \\ \hline
    \multicolumn{1}{|c|}{19.7}   & \multicolumn{1}{c|}{2.25}      & \multicolumn{1}{c|}{24.7}   & \multicolumn{1}{c|}{2.25}      & \multicolumn{1}{c|}{21.2}   & 2.75      \\ \hline
    \multicolumn{1}{|c|}{19.7}   & \multicolumn{1}{c|}{2.00}      & \multicolumn{1}{c|}{26.2}   & \multicolumn{1}{c|}{2.00}      & \multicolumn{1}{c|}{22.7}   & 2.50      \\ \hline
    \multicolumn{1}{|c|}{19.7}   & \multicolumn{1}{c|}{1.75}      & \multicolumn{1}{c|}{28.6}   & \multicolumn{1}{c|}{1.75}      & \multicolumn{1}{c|}{24.7}   & 2.25      \\ \hline
    \multicolumn{1}{|c|}{19.7}   & \multicolumn{1}{c|}{1.50}      & \multicolumn{1}{c|}{31.1}   & \multicolumn{1}{c|}{1.50}      & \multicolumn{1}{c|}{26.6}   & 2.00      \\ \hline
    \multicolumn{1}{|c|}{19.7}   & \multicolumn{1}{c|}{1.25}      & \multicolumn{1}{c|}{33.6}   & \multicolumn{1}{c|}{1.25}      & \multicolumn{1}{c|}{29.6}   & 1.75      \\ \hline
    \multicolumn{1}{|c|}{19.7}   & \multicolumn{1}{c|}{1.00}      & \multicolumn{1}{c|}{35.5}   & \multicolumn{1}{c|}{1.00}      & \multicolumn{1}{c|}{31.6}   & 1.50      \\ \hline
    \multicolumn{1}{|c|}{20.2}   & \multicolumn{1}{c|}{0.75}      & \multicolumn{1}{c|}{36.5}   & \multicolumn{1}{c|}{0.90}      & \multicolumn{1}{c|}{34.5}   & 1.25      \\ \hline
    \multicolumn{1}{|c|}{21.2}   & \multicolumn{1}{c|}{0.50}      & \multicolumn{1}{c|}{37.0}   & \multicolumn{1}{c|}{0.80}      & \multicolumn{1}{c|}{37.5}   & 1.00      \\ \hline
    \multicolumn{1}{|c|}{22.7}   & \multicolumn{1}{c|}{0.35}      & \multicolumn{1}{c|}{37.0}   & \multicolumn{1}{c|}{0.70}      & \multicolumn{1}{c|}{39.5}   & 0.75      \\ \hline
    \multicolumn{1}{|c|}{32.1}   & \multicolumn{1}{c|}{0.25}      & \multicolumn{1}{c|}{37.5}   & \multicolumn{1}{c|}{0.60}      & \multicolumn{1}{c|}{}       &           \\ \hline
    \multicolumn{1}{|c|}{}       & \multicolumn{1}{c|}{}          & \multicolumn{1}{c|}{38.0}   & \multicolumn{1}{c|}{0.50}      & \multicolumn{1}{c|}{}       &           \\ \hline
    \multicolumn{1}{|c|}{}       & \multicolumn{1}{c|}{}          & \multicolumn{1}{c|}{39.5}   & \multicolumn{1}{c|}{0.45}      & \multicolumn{1}{c|}{}       &           \\ \hline
    \multicolumn{3}{|c|}{$\Delta V = 0.05 $cm^3$}                                                     & \multicolumn{3}{c|}{\Delta P = 0.3}                     \\ \hline
  \end{tabular}
\end{table}

\begin{table}[H]
\centering
\caption{Presiones de vapor para diferentes temperaturas}
  \begin{tabular}{|c|c|}
  \hline
    T(°C)                  & P(atm)           \\ \hline
    20.5                   & 19.7             \\ \hline
    25.9                   & 22.7             \\ \hline
    29.6                   & 24.7             \\ \hline
    35.4                   & 28.6             \\ \hline
    41.3                   & 32.6             \\ \hline
    $\Delta_{esc} T = 0.2$°C & $\Delta P = 0.3 atm$ \\ \hline
  \end{tabular}
\end{table}
\subsection{Datos para la representación gráfica}
\begin{table}[H]
  \centering
  \caption{Datos para la representación gráfica}
  \begin{tabular}{|c|c|c|c|}
    \hline
    $x=1/T$ & $\Delta x$ & $y=\ln P$ & $\Delta y$ \\ \hline
    0.0488  & 0.0005     & 2.98     & 0.01       \\ \hline
    0.0386  & 0.0003     & 3.12     & 0.01       \\ \hline
    0.0338  & 0.0002     & 3.21     & 0.01       \\ \hline
    0.0282  & 0.0002     & 3.35     & 0.01       \\ \hline
    0.0242  & 0.0001     & 3.48     & 0.01       \\ \hline

  \end{tabular}
\end{table}
\subsection{Representación Gráfica}
\pagebreak

\section{Ajuste de datos}
Para calcular los parámetros de ajuste, utilizaremos que:
\begin{table}[H]
\caption{Datos para calcular los parámetros de ajuste}
\begin{tabular}{|l|l|l|l|l|l|}
\hline
n & $\Sigma x$ & $\Sigma y$ & $\Sigma x^2$ & $\Sigma y^2$ & $\Sigma x y$ \\ \hline
5 & $1.65 \times 10^{-2} $    & 16.15      & 0.00639472   & 52.30516323  & 0.5533079455 \\ \hline
\end{tabular}
\end{table}

Con esto, si escribimos $\ln(P) = \frac{a}{T} + b$, tendremos los parámetros dados por:
\begin{equation}
  a = \frac{n\sum x y - \sum x \sum y}{n \sum x^2 - \left(\sum x\right)^2}= -2791\text{K}
\end{equation}
\begin{equation}
  b = \frac{\sum y \sum x^2 - \sum x \sum x y}{n\sum x^2 - \left(\sum x \right)^2}= 10.57
\end{equation}
$$R^2=\frac{\left(\sum x y -\sum x \sum y\right)^2}{\left(n\sum x^2 - \left(\sum x\right)^2\right)\left(n\sum y^2 - \left(\sum y\right)^2\right)} = 0.99961$$ 


\section{Cálculo de Errores}
Para regresiones lineales, tenemos dos fuentes de error; el estadístico y el instrumental.

El primero calculamos con:

$$\Delta_{est} a= a\sqrt{\frac{R_1^{-2}-1}{n-2}} =25.2\text{K}$$ 
$$\Delta_{est} b = \Delta_{est} a\sqrt{\frac{\sum_i x_i^2}{n}} = 0.083$$
Y el segundo se calcula con:

  $$\Delta_{inst} a = \sqrt{\sum_j\left[ \frac{n\cdot x_j - \sum_ix_i}{n\sum_i x_i^2-\left(  \sum_i  x_i\right)^2} \right]^2 \cdot \Delta y_j^2} =72.6 \text{K}$$ 

  $$\Delta_{inst} b =\sqrt{\sum_j\left[ \frac{\sum x_i^2 -x_j \sum_i x_i}{n\sum_i x_i^2-\left(  \sum_i  x_i\right)^2} \right]^2 \cdot \Delta y_j^2} = 0.27 $$

Finalmente, se puede calcular el error total por la regla de cuadratura:

\begin{equation}
  \Delta a = \sqrt{\Delta_{est} a^2 + \Delta_{inst} a^2} = 72.5\text{K}
\end{equation}
\begin{equation}
  \Delta b = \sqrt{\Delta_{est} b^2 + \Delta_{inst} b^2} = 0.28
\end{equation}

O sea, tenemos que:

\begin{equation}
  \label{a}
  a = (-2.23\pm 0.07) 10^{3} \text{K}
\end{equation}
\begin{equation}
  \label{b}
  b = 10.6\pm 0.3
\end{equation}
Si comparamos la ecuación \eqref{final} con los parámetros en las ecuaciones \eqref{a} y \eqref{b}, podemos concluir que:

$$a = -\frac{\Delta h_v}{R}$$
$$b = \frac{\Delta h_v}{ T_b R} = \frac{-a}{T_b}$$

O sea,

$$\Delta h_V  = - a R = 18.5 \text{kJ/mol}$$
$$T_b = \frac{-a}{b} = 210 \text{K} = -63 \text{°C}$$ 

Finalmente, podemos utilizar nuestros parámetros para estimar la presión crítica, $P_{c}$ con la temperatura crítica que obtuvimos, $T_c = 319.9 K$:

$P_c = e^{\frac{a}{T_c} + b} = 40 \text{atm}$
Y, utilizando la fórmula estándar para propagación de errores, $\Delta f(x_1, \ldots, x_n) = \sqrt{\sum_i \frac{\partial f}{\partial x_i}}$, tendremos que:
$\Delta \Delta h_V = 0.6 \text{kJ/mol}$
$$\Delta T_b = 9 \text{K}$$
$$\Delta P_c = 10 \text{atm}$$
\section{Resultados}

\begin{tcolorbox}
  \begin{equation}
    \Delta h_V  = (18.5 \pm 0.6)\text{kJ/mol}
  \end{equation}
  \begin{equation}
    T_b = \frac{-a}{b} = (210 \pm 9) \text{K} = (-63 \pm 9) \text{°C}
  \end{equation}
  \begin{equation}
    P_c = 40\pm 10 \text{atm}
  \end{equation}
\end{tcolorbox}

\section{Conclusiones}
Para el cálculo de la entalpía molar de vaporización y de la temperatura de ebullición normal, tuvimos resultados bastante precisos y compatibles con las expectativas (en orden de magnitud). Para el valor de la temperatura crítica, tuvimos un resultado coherente con el obtenido experimentalmente (de aproximadamente 37 atm), pero bastante impreciso, con un error relativo de 25\%.




\end{document}

