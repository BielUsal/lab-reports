%Every piece of package I've acumulated over the last years
%%%%%%%%%%%%%%%%%%%%%%%%%%%%%%%%%%%%%%%%%%%%%%%%%%%%%%%%%%%%%%%%%%%%%%%%%%%%%%%%%%%%%%%%%%%%%%
\documentclass[a4paper,12pt]{article}
\usepackage[utf8]{inputenc}
\usepackage{imakeidx}
\usepackage{graphicx}
\usepackage{float}
\usepackage{amsmath}
\usepackage[backend=bibtex,style=verbose]{biblatex}
\bibliography{bibliography}
\usepackage{csquotes}
\usepackage{tcolorbox}
\usepackage{multirow}
\usepackage{caption}
\usepackage{afterpage}
\usepackage[margin=1in]{geometry}
%%%%%%%%%%%%%%%%%%%%%%%%%%%%%%%%%%%%%%%%%%%%%%%%%%%%%%%%%%%%%%%%%%%%%%%%%%%%%%%%%%%%%%%%%%%%%%
\begin{document}

\title{Practica 1: Simulaciones analógicas y técnicas numéricas}
\author{D'Andrade Furlanetto, Gabriel}
\maketitle

\section{Fundamento físico y objetivos}
Problemas electrostáticos se pueden hacer, en general, de dos maneras distintas: Por integración (Ley de Gauss o de Coulomb), que solo es factible en problemas dotados de mucha simetría, o por medio de ecuaciones diferenciales parciales (con la Ecuación de Poisson para el potencial) para todo lo demás. Particularmente, las ecuaciones diferenciales de la electrostática son todas elípticas y, por lo tanto, tienen solución determinada unicamente por la ecuacion y las condiciones de borde apropriada, o sea, son un problema de Condiciones de Borde (BVP)\footnote{Para más información sobre el tema, \cite[]{PDEs}}.

Cada uno de los problemas que trataremos van a ser diferentes condiciones de borde para la ecuación de Laplace:
$$\vec{\nabla}^2 \phi = 0 $$

Vamos a resolver el problema de dos maneras no-analíticas distintas: Primeramente, resolveremos un problema equivalente con corrientes estacionarias, despues, utilizaremos técnicas numéricas estandar de elemento finito. 

Para justificar la equivalencia del problema equivalente, solo necesitamos saber que la corriente es estacionaria, o sea, $\vec{\nabla} \cdot \vec{J} = 0$, y que el material es óhmico, $\vec{E} = \sigma \vec{J}$. Eso nos garantiza que, como $\vec{E} = -\vec{\nabla} \phi$, $\vec{\nabla}^2 \phi = \vec{\nabla} \cdot \sigma \vec{J}=0$, o sea, se verifica la ecuación de Laplace con condiciones de borde igual para $\vec{E}$ y $\vec{J}$.
\section{Realización practica}
\subsection{Problema 1}
\section{Comentarios}

\end{document}

