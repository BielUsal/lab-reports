%Every piece of package I've acumulated over the last years
%
%
\documentclass[a4paper,12pt]{article}
\usepackage[utf8]{inputenc}
\usepackage{imakeidx}
\usepackage{graphicx}
\usepackage{float}
\usepackage{amsmath}
\usepackage[backend=bibtex,style=verbose]{biblatex}
\bibliography{bibliography}
\usepackage{csquotes}
\usepackage{tcolorbox}
\usepackage{multirow}
\usepackage{caption}
\usepackage{afterpage}
%End of packages
%
%
%
\begin{document}

\title{Linear Colisions}
\author{Gabriel D'Andrade Furlanetto}
\maketitle
\pagebreak 
\abstract{In this paper, we analyze the collision of two objects experiencing minimal dissipative forces. We begin by doing a detailed study of the dissipative force present in our system, which is then followed up by a series of collisions under different initial conditions, different ellasticity conditions, and finally under different masses, and calculate relevant coefficients and relations for each of them. In doing this, we are able to analyze how well momentum and kinectic energy are conserved.}
\section{Introduction}

\subsection{Objectives}

\subsection{Theorethical Introduction}

\section{Experimental Procedure}

\subsection{Measurements}

\section{Results}

\section{Final Discussion}

\end{document}
